\documentclass[12pt]{amsart}
\usepackage[T1]{fontenc}
%\hyphenation{Sig-ur{\dh}s-son}
\begin{document}
\noindent \textbf{Speaker:} Peter May \\
\textbf{Title}: Parametrized homotopy theory \\
\indent \indent \indent (Joint work with J\'ohann Sigur{\dh}sson)

\medskip

\begin{center} ABSTRACT \end{center}
  What is parametrized homotopy theory?  The
homotopy theory of objects (spaces or spectra, perhaps
equivariant)  over a given base space.  What is it good for? All
homotopical work that involves bundles and fibrations. Using
spectra, it gives a context where one can apply the power of
stable homotopy theory and still remember such basic unstable
structure as fundamental groups.  Is it a routine generalization
of ordinary homotopy theory? No!!!. There are major conceptual and
technical difficulties that have no precursors in earlier work.
 The force of the theory
comes  from base change functors that relate objects over
different  base spaces. Pullback base change functors have both
left and  right adjoints, which are used in all applicaitons. You
cannot  get both adjunctions on homotopy categories by just using
Quillen model category theory. More serious work is needed (red
flag waving).
  Even in the parts  of the theory in which model category theory works, it doesn't work as
one would expect. There is an obviously ``right'' naive model
 structure on spaces over $B$, but it is nonobviously useless.
  Similar problems arise in sheaf theoretic contexts.
 While  our
solutions to these and related problems are special to the
topological context, they might serve as a guide to anyone
interested in modernizing the analogous foundations in algebraic
geometry.  Applications?  A fiberwise duality  theorem allows
fiberwise recognition of dualizable  and invertible parametrized
spectra.  Application of the  formal theory of duality (in
symmetric monoidal categories) in the parametrized setting gives a
conceptual construction and analysis of transfer maps in the
nonparametrized setting.  In the equivariant world, the theory
gives a simple conceptual proof  of a parametrized Wirthm\"uller
isomorphism that calculates the right adjoint to base change along
suitable maps in terms of a shift of the left adjoint.
 The Adams isomorphism relating orbit and fixed  point
spectra is a direct consequence. This is a fundamental result in
equivariant stable homotopy theory whose original (Lewis--May)
proof is impenetrable.  Work in progress: this is the definitively
right context in which to finally understand equivariant
Poincar\'e duality.
\end{document}
