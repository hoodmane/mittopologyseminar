\documentclass{slides}

\usepackage{times}

\setlength{\voffset}{-2cm}
%\setlength{\hoffset}{-2cm}
\setlength{\textwidth}{165mm}
\setlength{\textheight}{258 mm}
\begin{document}

\begin{center}

{\fontsize {54pt}{40pt}\selectfont

\textrm{
{\textbf{MIT Topology Seminar}}}
}\\
\vspace{1cm}
{\large\textrm{\emph{Monday, September 26, 4:30pm\\MIT Room 2-142}}}\\
\vspace{1cm}
\textrm{{\LARGE Doug Ravenel  \\[.5cm](MIT)}}\\
\vspace{1cm} %\bigskip\bigskip %\bigskip
\textrm{speaking on}\\ %\bigskip %\bigskip %\bigskip
\vspace{5mm}
\textrm{{\LARGE Toward Higher Chromatic Analogs of tmf}}\\
\end{center}
\vspace{1cm}
{\small
Abstract:  This talk is closely related to Mark Behrens' talk of 2 weeks ago.
TMF is a spectrum constructed using elliptic curves, and the link between
elliptic curves and stable homotopy theory is the theory of 1-dimensional
formal group laws.  The height of a FGL attached to an elliptic curve is at
most 2, so TMF gives infomration about $v_2$-periodic phenomena, but does not
take us deeper into the chromatic tower.  A curve of genus $g$ has a
$g$-dimensional FGL (the formal completion of it Jacobian) attached to it, and
there are examples where this FGL is known to a
1-dimensional summand of larger height.  A suitable moduli space of such curves
could conceivably lead to an analog of tmf.

\textrm{Contact email: } \texttt{mikehill@math}, \texttt{tgerhard@math},
\texttt{hrm@math}\\
\textrm{URL: } \texttt{www-math.mit.edu/topology}
}
\end{document}








