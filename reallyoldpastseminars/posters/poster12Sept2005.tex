\documentclass{slides}

\usepackage{times}

\setlength{\voffset}{-2cm}
%\setlength{\hoffset}{-2cm}
\setlength{\textwidth}{165mm}
\setlength{\textheight}{258 mm}
\begin{document}

\begin{center}

{\fontsize {54pt}{40pt}\selectfont

\textrm{
{\textbf{MIT Topology Seminar}}}
}\\
\vspace{1cm}
{\large\textrm{\emph{Monday, September 12, 4:30pm\\MIT Room 2-142}}}\\
\vspace{1cm}
\textrm{{\LARGE Mark Behrens  \\[.5cm](MIT)}}\\
\vspace{1cm} %\bigskip\bigskip %\bigskip
\textrm{speaking on}\\ %\bigskip %\bigskip %\bigskip
\vspace{5mm}
\textrm{{\LARGE Cohomology Theories Associated to Shimura Varieties}}\\
\end{center}
\vspace{1cm}
{\small
Abstract:  I will discuss some joint work (in progress) with Tyler Lawson.  I
will briefly motivate the talk by describing how to use modular forms
and the building for $GL_2$ to capture a piece of the second chromatic
layer of the homotopy groups of spheres.  I will then describe how a
similar construction (using the building for $GL_n$) may be considered
in connection with chromatic level n - where elliptic curves are
replaced with certain abelian varieties with complex multiplication.

\textrm{Contact email: } \texttt{mikehill@math}, \texttt{tgerhard@math},
\texttt{hrm@math}\\
\textrm{URL: } \texttt{www-math.mit.edu/topology}
}
\end{document}








