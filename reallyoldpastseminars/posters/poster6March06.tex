\documentclass{slides}

\usepackage{times}
%\usepackage{amssymb}
%\usepackage{amsmath}
\usepackage{bbold}
\setlength{\voffset}{-2cm}
%\setlength{\hoffset}{-2cm}
\setlength{\textwidth}{165mm}
\setlength{\textheight}{258 mm}
\begin{document}

\begin{center}

{\fontsize {54pt}{40pt}\selectfont

\textrm{
{\textbf{MIT Topology Seminar}}}
}\\
\vspace{.5cm}
{\large\textrm{\emph{Monday, March 6, 4:30pm\\MIT Room 2-142}}}\\
\vspace{.5cm}
\textrm{{\Large Georg Biedermann  \\[.5cm](University of Western Ontario)}}\\
\vspace{.5cm} %\bigskip\bigskip %\bigskip
\textrm{speaking on}\\ %\bigskip %\bigskip %\bigskip
\vspace{5mm}
\textrm{{\Large Model structures for Goodwillie calculus}}
\\
\end{center}
\vspace{.5cm}
{\small
Abstract:  The category of small covariant functors from simplicial sets to
 simplicial sets supports the projective model structure. In this paper
 we construct various localizations of the projective model structure and
 also give a variant for functors from simplicial sets to spectra. We
 apply these model categories in the study of calculus of functors,
 namely for classification of polynomial and homogeneous functors.
 Finally we show that the n-th derivative induces a Quillen map between
 the n-homogeneous model structure on small functors from pointed
 simplicial sets to spectra and the category of spectra with
 $\Sigma_n$-action. We consider also a finitary version of the
 n-homogeneous model structure and the n-homogeneous model structure on
 functors from pointed finite simplicial sets to spectra. In these two
 cases the above Quillen map becomes a Quillen equivalence. This improves
 the classification of finitary homogeneous functors by Goodwillie.


\textrm{Contact email: } \texttt{mikehill@math.mit.edu}, 
\\ \texttt{tgerhard@math.mit.edu}, \texttt{hrm@math.mit.edu}\\
\textrm{URL: } \texttt{www-math.mit.edu/topology}
}
\end{document}








