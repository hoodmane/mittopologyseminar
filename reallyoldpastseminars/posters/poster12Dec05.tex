\documentclass{slides}

\usepackage{times}

\setlength{\voffset}{-2cm}
%\setlength{\hoffset}{-2cm}
\setlength{\textwidth}{165mm}
\setlength{\textheight}{258 mm}
\begin{document}

\begin{center}

{\fontsize {54pt}{40pt}\selectfont

\textrm{
{\textbf{MIT Topology Seminar}}}
}\\
\vspace{.7cm}
{\large\textrm{\emph{Monday, December 12, 4:30pm\\MIT Room 2-142}}}\\
\vspace{.7cm}
\textrm{{\LARGE Kiyoshi Igusa  \\[.5cm]\large(Brandeis University)}}\\
\vspace{1cm} %\bigskip\bigskip %\bigskip
\textrm{speaking on}\\ %\bigskip %\bigskip %\bigskip
\vspace{5mm}
\textrm{{\LARGE Axiomatic higher torsion invariants and its consequences}}\\
\end{center}
\vspace{1cm}
{\tiny Abstract: This talk is about the axiomatic approach to higher Franz-
Reidemeister (FR) torsion.
We consider smooth manifold bundles $M\to E\to B$ satisfying certain
conditions (e.g., $B$ simply connected is sufficient for all cases).
For these bundles there are several real characteristic classes $\tau
(E)\in H^{4k}(B;R)$: higher FR torsion, analytic torsion classes,
higher Dwyer-Weiss-Williams (DWW) classes and tautological (Miller-
Morita-Mumford) classes.
However, for fixed $k$ and fixed parity of the dimension of $M$ there
is only one characteristic class (up to a scalar multiple) satisfying
two axioms.

Using this theorem, old theorems become clearer and there are some
new results.
So far, only the FR-torsion and tautological classes are known to
satisfy the axioms. So they are proportional (old result). Sebastian
Goette and I have some new results which imply equality with DWW
torsion in some cases. Finally, I will attempt to extend axiomatic
torsion to the equivariant case.

\textrm{Contact email: } \texttt{mikehill@math.mit.edu},\\
 \texttt{tgerhard@math.mit.edu},
\texttt{hrm@math.mit.edu}\\
\textrm{URL: } \texttt{www-math.mit.edu/topology}
}
\end{document}








