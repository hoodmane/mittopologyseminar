\documentclass{slides}

\usepackage{times}

\setlength{\voffset}{-2cm}
%\setlength{\hoffset}{-2cm}
\setlength{\textwidth}{165mm}
\setlength{\textheight}{258 mm}
\begin{document}

\begin{center}

{\fontsize {54pt}{40pt}\selectfont

\textrm{
{\textbf{MIT Topology Seminar}}}
}\\
\vspace{1cm}
{\large\textrm{\emph{Monday, November 21, 4:30pm\\MIT Room 2-142}}}\\
\vspace{1cm}
\textrm{{\LARGE Kevin Costello  \\[.5cm](University of Chicago)}}\\
\vspace{1cm} %\bigskip\bigskip %\bigskip
\textrm{speaking on}\\ %\bigskip %\bigskip %\bigskip
\vspace{5mm}
\textrm{{\LARGE Constructing the Gromov-Witten potential associated to a topological conformal field theory }}\\
\end{center}
\vspace{1cm}
{\tiny
Abstract: This talk is about the relation between the moduli spaces of Riemann surfaces, Batalin-Vilkovisky algebras and linear symplectic geometry.  I'll show how inside the uncompactified moduli spaces of Riemann surfaces one can find, in a canonical way up to homotopy, a chain playing the role of the fundamental class of the Deligne-Mumford spaces.  This is not closed, but satisfies a certain BV master equation, for a BV algebra structure introduced by Sen and Zwiebach.  This construction allows us to construct the analog of the Gromov-Witten potential associated to a TCFT.  This is a state in a certain Fock space.

\textrm{Contact email: } \texttt{mikehill@math.mit.edu},\\
 \texttt{tgerhard@math.mit.edu},
\texttt{hrm@math.mit.edu}\\
\textrm{URL: } \texttt{www-math.mit.edu/topology}
}
\end{document}








