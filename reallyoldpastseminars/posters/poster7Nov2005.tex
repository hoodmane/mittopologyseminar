\documentclass{slides}

\usepackage{times}

\setlength{\voffset}{-2cm}
%\setlength{\hoffset}{-2cm}
\setlength{\textwidth}{165mm}
\setlength{\textheight}{258 mm}
\begin{document}

\begin{center}

{\fontsize {54pt}{40pt}\selectfont

\textrm{
{\textbf{MIT Topology Seminar}}}
}\\
\vspace{.5cm}
{\large\textrm{\emph{Monday, November 7, 4:30pm\\MIT Room 2-142}}}\\
\vspace{.5cm}
\textrm{{\LARGE Kari Ragnarsson}}  \\
\vspace{.5cm} %\bigskip\bigskip %\bigskip
\textrm{speaking on}\\ %\bigskip %\bigskip %\bigskip
\vspace{5mm}
\textrm{{\LARGE A Segal conjecture for p-completed classifying spaces}}\\
\end{center}
\vspace{1cm}
{\tiny
Abstract: As was predicted by Adams and Miller, and shown by
Lewis-May-McClure, one consequence of Carlsson's solution of the Segal
conjecture is the description of the group $\{BG,BH\}$ of homotopy classes of
stable maps between classifying spaces of finite groups G and H as the
completion of $A(G,H)$ at the augmentation ideal $I(G)$ of the Burnside ring
$A(G)$. (Here $A(G,H)$ denotes the Grothendieck group completion of the monoid
of isomorphism classes of finite (G x H)-sets such that the induced
H-action is free, and $A(G)$ can be regarded as the special case where H is
the trivial group.) Unfortunately such completions are very difficult to
calculate in general. However, Lewis-May showed that in the special case
where G is a p-group, $I(G)$-adic completion agrees with p-adic completion.

In this talk I will illustrate how the simplification of Lewis-May can be
extended to general finite groups G, but at the cost of p-completing
classifying spaces. More precisely, let $A_p(G,H)$ denote the submodule of
$A(G,H)$ generated by those $(G x H)$-sets whose isotropy groups under the
$G$-action are p-groups. I will show that $\{BG^{}_p,BH\}$ is the p-completion of
$A_p(G,H)$. Since $BG$ is the wedge sum of the $BG^{}_p$ for different primes
$p$, we can collect these results to get a new and simple description of
$\{BG,BH\}$.

\textrm{Contact email: } \texttt{mikehill@math}, \texttt{tgerhard@math},
\texttt{hrm@math}\\
\textrm{URL: } \texttt{www-math.mit.edu/topology}
}
\end{document}








