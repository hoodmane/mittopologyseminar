\documentclass{slides}

\usepackage{times}
%\usepackage{amssymb}
%\usepackage{amsmath}
\usepackage{bbold}
\setlength{\voffset}{-2cm}
%\setlength{\hoffset}{-2cm}
\setlength{\textwidth}{165mm}
\setlength{\textheight}{258 mm}
\begin{document}

\begin{center}

{\fontsize {54pt}{40pt}\selectfont

\textrm{
{\textbf{MIT Topology Seminar}}}
}\\
\vspace{.5cm}
{\large\textrm{\emph{Monday, February 13, 4:30pm\\MIT Room 2-142}}}\\
\vspace{.5cm}
\textrm{{\Large Eric Rains  \\[.5cm](UC Davis)}}\\
\vspace{.5cm} %\bigskip\bigskip %\bigskip
\textrm{speaking on}\\ %\bigskip %\bigskip %\bigskip
\vspace{5mm}
\textrm{{\Large The cohomology ring of the real locus of ${\bar M}_{0,n}$}}\\
\end{center}
\vspace{.5cm}
{\small
Abstract:  The moduli space $M_{0,n}$ (the set of equivalence classes of
 n-\,\,tuples of distinct points on the projective line under simultaneous
 linear fractional transformations) has a natural compactification to a
 smooth projective scheme ${\bar M}_{0,n}$.  Since this scheme is defined
 over $\mathbb{Z}$, its real locus is a smooth (in general nonorientable)
 manifold.  The rational cohomology algebra of this manifold has a number
 of interesting properties, most notably the fact that its Poincar\'e
 polynomial factors completely (in sharp contrast to the corresponding
 complex manifold).  I'll discuss recent work with Etingof, Henriques,
 and Kamnitzer deriving this Poincar\'e polynomial, as well as an
 explicit presentation and basis of the cohomology algebra.


\textrm{Contact email: } \texttt{mikehill@math.mit.edu}, 
\\ \texttt{tgerhard@math.mit.edu}, \texttt{hrm@math.mit.edu}\\
\textrm{URL: } \texttt{www-math.mit.edu/topology}

\end{document}








