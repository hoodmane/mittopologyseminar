\documentclass{slides}

\usepackage{times}

\setlength{\voffset}{-2cm}
%\setlength{\hoffset}{-2cm}
\setlength{\textwidth}{165mm}
\setlength{\textheight}{258 mm}
\begin{document}

\begin{center}

{\fontsize {54pt}{40pt}\selectfont

\textrm{
{\textbf{MIT Topology Seminar}}}
}\\
\vspace{1cm}
{\large\textrm{\emph{Monday, February 23, 4:30pm\\MIT Room 2-131}}}\\
\vspace{1cm}
\textrm{{\LARGE Ruth Charney  \\[.5cm](Brandeis University)}}\\
\vspace{1cm} %\bigskip\bigskip %\bigskip
\textrm{speaking on}\\ %\bigskip %\bigskip %\bigskip
\vspace{5mm}
\textrm{{\LARGE A new look at the affine braid groups}}\\
\end{center}
\vspace{1cm}
{\small
Abstract:  Affine braid groups are the braid groups corresponding to 
the Euclidean Coxeter groups of type $\tilde A_n$. They can be 
described as braids on a cylinder. These groups embed in a standard 
braid group in a nice way.  We exploit this embedding to get new 
information about the affine braid groups. In particular, we prove that 
the complex hyperplane complement associated to the affine Coxeter 
group is a K($\pi$,1) space for the affine braid group. (joint work 
with D. Peifer)

\textrm{Contact email: } \texttt{karir@math.mit.edu}, 
\texttt{hrm@math.mit.edu}\\
\textrm{URL: } \texttt{www-math.mit.edu/topology}
}
\end{document}








