\documentclass{article}

\usepackage{cmbright}
\usepackage{scalefnt}
\usepackage{anyfontsize}
\usepackage{textcomp}
\usepackage{amssymb}
\usepackage{amsmath}
\usepackage[utf8]{inputenc}


\DeclareMathOperator{\N}{\mathbb{N}}
\DeclareMathOperator{\Z}{\mathbb{Z}}
\DeclareMathOperator{\Q}{\mathbb{Q}}
\DeclareMathOperator{\R}{\mathbb{R}}
\DeclareMathOperator{\C}{\mathbb{C}}
\DeclareMathOperator{\H}{\mathbb{H}}
\DeclareMathOperator{\A}{\mathbb{A}}
\let\infinity\infty

\providecommand{\mparwidth}{1in}
\providecommand{\mtop}{0.5in}
\providecommand{\mbottom}{0.5in}
\providecommand{\mleft}{0.5in}
\providecommand{\mright}{0.5in}
\usepackage[top = \mtop, bottom = \mbottom, left = \mleft, right=\mright, marginparwidth=\mparwidth]{geometry}
\pagenumbering{gobble}

\begin{document}

\begin{center}

{\scalefont{5}Topology Seminar}
\medskip\vspace{2cm}

{\scalefont{3.6}{\bf Soren Galatius}}\\\vspace{20pt}
{\scalefont{2}of Stanford University will be speaking on}\\\vspace{30pt}
{\scalefont{3.6}hocolim decomposition of compactified moduli space}\\\vspace{20pt}
{\scalefont{2}on Monday, September 15 at 4:30\,pm in\\\vspace{3pt}MIT Room 2-131}\\
\end{center}

\vfill

\begin{center}
\begin{tabular}{p{.9\textwidth}}
\scalefont{1.5}

The moduli space of Riemann surfaces $M$ is a classifying space for families of Riemann surfaces. It has a compactification $\bar M$, which is a classfying space for families of modal Riemann surfaces. A nodal Riemann surface is allowed to have singularities which look like the solutions to $zw=0$ in complex $2$-space. I will describe how to decompose $\bar M$ as a homotopy colimit of spaces which look more like $M$. Then I will use this to study part of the homology of $\bar M$, using what is known about the homology of $M$.
\end{tabular}
\end{center}

\vfill

\centerline{\scalefont{1.2}
For information, write: \texttt{hood@mit.edu}
}
\vfill
\end{document}

% vim:ft=tex
