\documentclass{article}

\usepackage{cmbright}
\usepackage{scalefnt}
\usepackage{anyfontsize}
\usepackage{textcomp}
\usepackage{amssymb}
\usepackage{amsmath}
\usepackage[utf8]{inputenc}


\DeclareMathOperator{\N}{\mathbb{N}}
\DeclareMathOperator{\Z}{\mathbb{Z}}
\DeclareMathOperator{\Q}{\mathbb{Q}}
\DeclareMathOperator{\R}{\mathbb{R}}
\DeclareMathOperator{\C}{\mathbb{C}}
\DeclareMathOperator{\H}{\mathbb{H}}
\DeclareMathOperator{\A}{\mathbb{A}}
\let\infinity\infty

\providecommand{\mparwidth}{1in}
\providecommand{\mtop}{0.5in}
\providecommand{\mbottom}{0.5in}
\providecommand{\mleft}{0.5in}
\providecommand{\mright}{0.5in}
\usepackage[top = \mtop, bottom = \mbottom, left = \mleft, right=\mright, marginparwidth=\mparwidth]{geometry}
\pagenumbering{gobble}

\begin{document}

\begin{center}

{\scalefont{5}Topology Seminar}
\medskip\vspace{2cm}

{\scalefont{3.6}{\bf Gon\c calo Tabuada}}\\\vspace{20pt}
{\scalefont{2}of Universidade Nova de Lisboa will be speaking on}\\\vspace{30pt}
{\scalefont{3.6}Non-commutative motives}\\\vspace{20pt}
{\scalefont{2}on Monday, March 29 at 4:30\,pm in\\\vspace{3pt}MIT Room 2-131}\\
\end{center}

\vfill

\begin{center}
\begin{tabular}{p{.9\textwidth}}
\scalefont{1.5}

In this talk I will describe the construction of the category of non-commutative motives $[1,2,3]$ in Drinfeld-Kontsevich's non-commutative algebraic geometry program. In the process, I will present the first conceptual characterization of Quillen's higher K-theory since Quillen's foundational work in the 70's. As an application, I will show how these results allow us to obtain for free the higher Chern character from K-theory to cyclic homology.

References:

[1] D.-C. Cisinski and G. Tabuada, Symmetric monoidal structure on
Non-commutative motives. Available at arXiv:1001.0228.

[2] D.-C. Cisinski and G. Tabuada, Non-connective K-theory via universal
invariants. Available at arXiv:0903.3717.

[3] G. Tabuada, Higher K-theory via universal invariants. Duke Math.
Journal, 145 (2008), no.1, 121-206.
2010/03/30,,,,Matthew Gelvin"
\end{tabular}
\end{center}

\vfill

\centerline{\scalefont{1.2}
For information, write: \texttt{hood@mit.edu}
}
\vfill
\end{document}

% vim:ft=tex
