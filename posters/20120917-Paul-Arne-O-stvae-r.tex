\documentclass{article}

\usepackage{cmbright}
\usepackage{scalefnt}
\usepackage{anyfontsize}
\usepackage{textcomp}
\usepackage{amssymb}
\usepackage{amsmath}
\usepackage[utf8]{inputenc}


\DeclareMathOperator{\N}{\mathbb{N}}
\DeclareMathOperator{\Z}{\mathbb{Z}}
\DeclareMathOperator{\Q}{\mathbb{Q}}
\DeclareMathOperator{\R}{\mathbb{R}}
\DeclareMathOperator{\C}{\mathbb{C}}
\DeclareMathOperator{\H}{\mathbb{H}}
\DeclareMathOperator{\A}{\mathbb{A}}
\let\infinity\infty

\providecommand{\mparwidth}{1in}
\providecommand{\mtop}{0.5in}
\providecommand{\mbottom}{0.5in}
\providecommand{\mleft}{0.5in}
\providecommand{\mright}{0.5in}
\usepackage[top = \mtop, bottom = \mbottom, left = \mleft, right=\mright, marginparwidth=\mparwidth]{geometry}
\pagenumbering{gobble}

\begin{document}

\begin{center}

{\scalefont{5}Topology Seminar}
\medskip\vspace{2cm}

{\scalefont{3.6}{\bf Paul Arne \O stv\ae r}}\\\vspace{20pt}
{\scalefont{2}of University of Oslo will be speaking on}\\\vspace{30pt}
{\scalefont{3.6}Motivic slices and the graded Witt ring}\\\vspace{20pt}
{\scalefont{2}on Monday, September 17 at 4:30\,pm in\\\vspace{3pt}MIT Room 2-131}\\
\end{center}

\vfill

\begin{center}
\begin{tabular}{p{.9\textwidth}}
\scalefont{1.5}

We compute the motivic slices of hermitian $K$-theory and higher Witt-theory. The corresponding slice spectral sequences relate motivic cohomology to hermitian $K$-groups and Witt groups, respectively. Using this we compute the hermitian $K$-groups of number fields, and (re)prove Milnor's conjecture on quadratic forms for fields of characteristic different from 2. Joint work with Oliver R\"ondigs.
\end{tabular}
\end{center}

\vfill

\centerline{\scalefont{1.2}
For information, write: \texttt{hood@mit.edu}
}
\vfill
\end{document}

% vim:ft=tex
